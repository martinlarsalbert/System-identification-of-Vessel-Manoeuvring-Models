\begin{frontmatter}

  %% \title{\tnoteref{t1,t2}}
   %%\tnotetext[t1]{This document is a collaborative effort.}
   %%\tnotetext[t2]{The second title footnote which is longer 
   %%    than the first one and with an intention to fill
   %%   in up more than one line while formatting.} 
  
   %%\title\tnoteref{t1,t2}}
   %%\tnotetext[t1]{This document is the results of the research
   %%   project funded by the National Science Foundation.}
   %%\tnotetext[t2]{The second title footnote which is a longer 
   %%   text matter to fill through the whole text width and 
   %%   overflow into another line in the footnotes area of the 
   %%   first page.}
  
  \author[1,2]{Martin Alexandersson\corref{cor1}%
    \fnref{fn1}}
  \ead{maralex@chalmers.se}
  
  \author[1]{Wengang Mao\fnref{fn2}}
  \ead{wengang.mao@chalmers.se}
  
  \author[1]{Jonas W Ringsberg\fnref{fn1,fn3}}
  \ead{jonas.ringsberg@chalmers.se}
  
  \cortext[cor1]{Corresponding author}
  %%\fntext[fn1]{This is the first author footnote.}
  %%\fntext[fn2]{Another author footnote, this is a very long footnote and
  %% it should be a really long footnote. But this footnote is not yet
 %%   sufficiently long enough to make two lines of footnote text.}
 %% \fntext[fn3]{Yet another author footnote.}
  
  \affiliation[1]{organization={Dept. of Mechanics and Maritime Sciences, 
                                Chalmers University of Technology},
                  addressline={Hörsalsvägen 7A}, 
                  city={Gothenburg},
  %               citysep={}, % Uncomment if no comma needed between city and postcode
                  postcode={41296}, 
                  state={Gothenburg},
                  country={Sweden}}
  
  \affiliation[2]{organization={SSPA Sweden AB},
                  addressline={Chalmers tvärgata 10}, 
                  postcode={41296}, 
                  postcodesep={}, 
                  city={Gothenburg,},
                  country={Sweden}}
  
  
  \begin{abstract}
  Different manoeuvring performance models have been investigated with their parameters identified by the proposed PIT method. In this paper, the robust VMM with proper parameters (hydrodynamic derivatives) is first chosen to describe a ship’s manoeuvring. To estimate the coefficients, this method can solve the data noise from model tests or full-scale measurements by combing the inverse dynamics regression and Extended Kalman Filter with a RTS smoother. The multicollinearity problems in the VMMs are addressed by introducing the thrust force models for different ship types. Two case study vessels with very different manoeuverability characteristics are used to demonstrate and validate the proposed method to build a VMM for their manoeuver.
  \end{abstract}
  
  \begin{keyword}
    Ship Manoeuvring, Parameter Identification, Inverse Dynamics, Extended Kalman Filter, RTS smoother, Multicollinearity
  \end{keyword}
  
  \end{frontmatter}